\documentclass{beamer}
% * <hannah.short@cern.ch> 2018-08-09T07:11:21.779Z:
%
% ^.
\usepackage[utf8]{inputenc}
\usepackage{multicol}
\usepackage{hyperref}
\usetheme{cern}
\usepackage{wrapfig}

%The CERN logo is legally protected. Please visit http://cern.ch/copyright for information on the terms of use of CERN content, including the CERN logo.

% The optional `\author` command defines the author and is displayed in the slide produced by the `\titlepage` command.
\author{Hannah Short (CERN), Sebastian Lopienski (CERN)}

% The optional `\title` command defines the title and is displayed in the slide produced by the `\titlepage` command.
\title{Introduction to Information Security}

% The optional `\subtitle` command will add a smaller title below the main one, and will not be displayed in any of the slides' footer.
\subtitle{CODATA School}

% The optional `\date` command will display a custom free text date on the all of the slides' footer. If omitted today's date will be used.
%\date{Monday, 1st January 2018}

\begin{document}

\frontcover

% The optional `\titlepage` command will create a slide with the presentation's title, subtitle and author.
\frame{\titlepage}

% The optional `\tableofcontents` command will automatically create a table of contents based om the sections.
\frame{\tableofcontents}

\begin{frame}{Lecturers}
These slides have been compiled by members of the CERN Computer Security Team based at CERN, the European Organisation for Nuclear Research.
\begin{center}
\begin{tabular}{ c c  }
 \includegraphics[width=0.2\linewidth]{lecturer1.png} & \includegraphics[width=0.2\linewidth]{Lecturer2.png}  \\ 
 Hannah Short & Sebastian Lopienski  \\  
\end{tabular}
\end{center}
\end{frame}

\begin{frame}{Course Objectives}
	\begin{itemize}
		\item Understand why Security is important for you as a Data Scientist
		\item Familiarise yourself with the basic principles of Information Security
	\end{itemize}
\textbf{Note:} \newline
\textit{}{If the slide title is in {\color{red}red}, the slide is considered an advanced topic}
\end{frame}

\section{Why Security?}
\frame{\sectionpage}

\begin{frame}{Why Security?}
	\begin{itemize}
		\item You are constantly exposed to reputational, financial and even physical risks online
		\item The aim is to \textbf{minimise your exposure to risk} through
        \begin{itemize}
        	\item Secure online activity
            \item Secure software design
        \end{itemize}
	\end{itemize}
\end{frame}

\begin{frame}{Safety vs Security}
\textbf{Safety} is about protecting from \textbf{accidental risks} 
\begin{itemize}
\item road safety
\item  air travel safety
\end{itemize}
\textbf{Security} is about mitigating risks of dangers
caused by \textbf{intentional, malicious actions} 
\begin{itemize}
\item homeland security
\item airport and aircraft security
\item information and computer security
\end{itemize}
\end{frame}

\begin{frame}{Why is security difficult?}
Security is as strong as the weakest link. There is no 100\% security!
\begin{center}
\includegraphics[width=0.8\linewidth]{link.png}
\end{center}
\end{frame}

\begin{frame}{What is risk?}
    \begin{itemize}
		\item Probability * impact
		\item Risks should be: Assessed, Prioritised, Mitigated, Avoided and finally Accepted
	\end{itemize}
    \begin{center} 
      \includegraphics[width=0.45\linewidth]{risk-matrix.png} 
    \end{center}
\end{frame}

\begin{frame}{Typical Threats}
\center 
But we're Scientists, surely we're not a target...! 
\end{frame}

\begin{frame}{Typical Threats}
  \begin{center}
		\includegraphics[width=0.65\linewidth]{lhc-attacked.png} \newline
        {\small \url{https://www.wired.com/2008/09/hackers-infiltr/} \par}
  \end{center}
\end{frame}

\begin{frame}{Typical Threats}
  \begin{center}
      \includegraphics[width=0.4\linewidth]{greek-attack.png}  \newline
  	 {\small \url{https://www.wired.com/2008/09/hackers-infiltr/} \par}
  \end{center}
\end{frame}

\begin{frame}{Attackers}
  \begin{center} 
      \includegraphics[width=1\linewidth]{attackers.png} 
    \end{center}
\end{frame}

\begin{frame}{Hacking as a Business}
\begin{center}
\includegraphics[width=0.8\linewidth]{google-bounty.png}
\end{center}
\end{frame}

\begin{frame}{Hacking as a Business}
\begin{center}
\includegraphics[width=0.8\linewidth]{facebook-bounty.png}
\end{center}
\end{frame}

\begin{frame}{Why Security - Summary}
\begin{itemize}
\item Security = mitigating risk of malicious actions
\item Science is an interesting target for bad guys/girls
\end{itemize}
\end{frame}

\section{Data Security Concepts}
\frame{\sectionpage}

\begin{frame}{Data Security Concepts}
At the heart of Security we have three key components:
	\begin{itemize}
		\item Technology
		\item Processes
        \item People
	\end{itemize}
\end{frame}

\begin{frame}{Technology}
We will come back to some of this in part 2 of our lecture course :) 
\end{frame}

\begin{frame}{Processes}
\textit{``Security is a process, not a product"} - Bruce Schneier
\end{frame}

\begin{frame}{Processes}
\begin{center}
\begin{tabular}{ |c|c| }
\hline
 \textbf{Security Measure} & \textbf{Requires}\\
\hline \hline
 Antivirus software & Virus signature Updates \\ \hline
 Monitoring systems & Checking, reacting to alarms \\  \hline
 Endpoint security & OS and software patching \\ \hline
 Security policies & Updating, enforcing \\
\hline
\end{tabular}
\end{center}
Risk management, vulnerability management, business continuity planning, security  development lifecycle etc... \textbf{these are ongoing processes, not one-off exercises.}
\end{frame}

\begin{frame}{Processes}
\begin{center}
\includegraphics[width=0.8\linewidth]{process1.png} 
\end{center}
\end{frame}

\begin{frame}{Processes}
Security solutions often degrade with time - they need to be verified periodically!
\begin{center}
\includegraphics[width=0.7\linewidth]{process2.png} 
\end{center}
\end{frame}

\begin{frame}{People}
	\begin{itemize}
		\item Have flawed risk perception
		\item Are bad at dealing with exceptions and rare cases
        \item Can't take correct security decisions
        \item Put too much trust in their computers
        \item Easily fall for social engineering
        \item Sometimes turn malicious
        \item Prefer convenience and bypass security measures
        \item Often make mistakes...
	\end{itemize}
\end{frame}

\begin{frame}{Risk Perception}
Is flying more dangerous than traveling by car?
\newline
\includegraphics[width=0.8\linewidth]{planecar.png}
\newline 
Are you more likely to be killed by a shark, a pig or a coconut?
\newline
\includegraphics[width=0.8\linewidth]{sharkpigcoco.png}
\end{frame}

\begin{frame}{Social Engineering}
\begin{center}
\includegraphics[width=0.4\linewidth]{socialengineering.png}\newline
{\small \url{https://www.smbc-comics.com}}
\end{center}
\end{frame}

\begin{frame}{Social Engineering}
  \begin{itemize}
    \item First the Social Engineer gathers information:
    \begin{itemize}
    	\item Public and semi public information; names, hierarchy, who's on holiday, project names etc
    \end{itemize}
    \item Armed with the information they:
     \begin{itemize}
    	\item Use \textbf{influence, persuasion} or \textbf{threat}
        \item Abuse people's \textbf{compassion, fear} or \textbf{greed}
        \item Exploit tendency to \textbf{trust} and help
    \end{itemize}
   	\item In order to gain unauthorised access to systems or information
  \end{itemize}
\end{frame}

\begin{frame}
\begin{center}
\includegraphics[width=0.9\linewidth]{ebay.png}
\end{center}
\end{frame}

\begin{frame}{Taking security decisions}
Users typically make poor security choices despite systems trying to protect them!
\includegraphics[width=1\linewidth]{security-decisions.png}
\end{frame}

\begin{frame}{Data Security Concepts - Summary}
\begin{itemize}
\item Processes must be ongoing, security degrades with time
\item People often provide the easiest way for an attacker to compromise the system 
\item Security is only as strong as the weakest link - don't lock the front door but leave the back door open!
\end{itemize}
\end{frame}

\section{Security Objectives}
\frame{\sectionpage}

\begin{frame}{Security Objectives}
	\begin{itemize}
		\item Confidentiality
		\item Integrity
        \item Availability
	\end{itemize}
\end{frame}


\begin{frame}{Security Objectives}
	\begin{itemize}
		\item \textbf{Confidentiality}
		\item Integrity
        \item Availability
	\end{itemize}
    Can the correct people access the data at the correct time?
	\linebreak
    \linebreak
    { \color{red} \textit{Security Tip: Pay attention to where your data is stored and how it is shared!} }
\end{frame}

\begin{frame}{Confidentiality}
\begin{itemize}
\item Your online identity is as valuable as your passport 
\item Your authorisation may be misused if it falls into the wrong hands
\end{itemize}
{ \color{red} \textit{Security Tip: Store your secrets safely, not in the public domain, e.g. github} }
\end{frame}

\begin{frame}{}
\begin{center}
\includegraphics[width=0.8\linewidth]{github-bitcoin.png}
\end{center}
\end{frame}

\begin{frame}{How bad can it be?}
\begin{itemize}
\item 5 minutes exposure
\item \$2,375
\item Plus it could have been avoided, Amazon has a service (IAM) to manage keys securely...
\end{itemize}
{\small \url{https://www.theregister.co.uk/2015/01/06/ dev_blunder_shows_github_crawling_with_keyslurping_bots/} \par}
\end{frame}

\begin{frame}{Security Objectives}
	\begin{itemize}
		\item Confidentiality
		\item \textbf{Integrity}
        \item Availability
	\end{itemize}
    Can we be sure that the data is reliable and hasn't been altered?
	\linebreak
    \linebreak
    { \color{red} \textit{Security Tip: Reduce the risk of impersonation, enable multi-factor authentication wherever possible!} }
\end{frame}


\begin{frame}{Security Objectives}
	\begin{itemize}
		\item Confidentiality
		\item Integrity
        \item \textbf{Availability}
	\end{itemize}
    Is the data available? Are our systems reliable?
	\linebreak
    \linebreak
    { \color{red} \textit{Security Tip: Keep backups!} }
\end{frame}

\begin{frame}{Identity}
A key part of ensuring Confidentiality, Integrity and Availability objectives are met! 
\begin{itemize}
\item How can you know who is using a system?
\item How can you ensure that only the correct people are using a system?
\end{itemize}
\end{frame}

\begin{frame}{Authentication and Authorisation}
\includegraphics[width=1\linewidth]{authn-authz.png}
\end{frame}

\begin{frame}{Authentication and Authorisation}
Authentication = Who am I? How can I prove my Identity?
\includegraphics[width=1\linewidth]{authn-authz.png}
\end{frame}

\begin{frame}{Authentication and Authorisation}
Authorisation = What am I able to do?
\includegraphics[width=1\linewidth]{authn-authz.png}
\end{frame}
\begin{frame}{Multifactor Authentication}
\begin{center}
\begin{tabular}{ |c|c|c| }
\hline
 \textbf{Factor} & \textbf{Description} & \textbf{Example}\\
\hline \hline
 1 & Something you know & Password, pin\\ \hline
 2 & Something you have & Phone,  token \\  \hline
 3 & Something you are & Fingerprint, iris scan \\ \hline
\end{tabular}
\end{center}
Which is most secure?
\end{frame}

\begin{frame}{Security Objectives - Summary}
\begin{itemize}
\item Key objectives: Confidentiality, Integrity and Availability
\item Consider disaster scenarios and plan for them 
\item Authentication and Authorisation are critical to meeting security objectives
\end{itemize}
\end{frame}

\section{Guidelines and Principles}
\frame{\sectionpage}

\begin{frame}{Security Measures}
Is this a good security measure?
\begin{center}
\includegraphics[width=0.6\linewidth]{flipflops.png}
\end{center}
\end{frame}

\begin{frame}{Security Measures}
\begin{itemize}
 \item What problem is it trying to solve? 
 \item Does it help? 
 \item Does it introduce new problems? 
 \item What are the costs?
 \end{itemize} 
 \begin{center}
	\includegraphics[width=0.4\linewidth]{flipflops.png} 
 \end{center}
\end{frame}

\begin{frame}{Security Measures}
How much security?
\begin{tabular}{c c}
\includegraphics[height = 3.5cm]{ostrich.jpg}
& \includegraphics[height = 3.5cm]{cat.jpg}
\end{tabular}
\centering
It's a balance of risk, usability and cost 
\end{frame}

\begin{frame}{Security Design Principles}
  \begin{itemize}
    \item Defense in depth
    \item Deny by default
    \item Least privilege principle
    \item Complex = insecure
    \item Security, not obscurity
  \end{itemize}
\end{frame}

\begin{frame}{Defense in depth}
How can you avoid a single point of failure? Where should you keep your assets?
\begin{center}
\includegraphics[width=0.7\linewidth]{castle.png}
\end{center}
\end{frame}

\begin{frame}{{\color{red}Deny by default}}
Use whitelisting rather than blacklisting
\begin{center}
\includegraphics[width=0.8\linewidth]{whitelisting.png}
\end{center}
\end{frame}

\begin{frame}{Least privilege principle}
“Need to know” basis: require, grant and use only the privileges that are really needed
\end{frame}

\begin{frame}{Complex = insecure}
Maintenance of complex code leads to vulnerabilities
\begin{center}
\includegraphics[width=0.6\linewidth]{complex.png} \newline
System calls in Apache
\end{center}
\end{frame}

\begin{frame}{Security by obscurity}
What is it? Hiding design or implementation details to gain security:
\begin{itemize}
\item e.g. hiding a DB server under a name different from “db”, etc.
\item e.g. keeping the encryption algorithm secret, instead of the key
\end{itemize}
\end{frame}

\begin{frame}{Security by obscurity}
The idea doesn’t work
\begin{itemize}
\item It’s difficult to keep secrets (e.g. source code gets stolen, Google indexes hidden pages...)
\item If security of a system depends on a secret that's revealed, the whole system is compromised
\item Secret algorithms, protocols etc. will not get reviewed, flaws won’t be spotted and fixed, less security
\end{itemize}
\centering
{ \color{red} \textit{Systems should be secure by design, not by obfuscation!} }
\end{frame}

\begin{frame}{Guidelines and Principles - Summary}
\begin{itemize}
\item Security is a balance of risk, usability and cost
\item The Security Design Principles discussed will help you prioritise security
\item Ensure Security Design Principles are included from the very beginning of a software project
\end{itemize}
\end{frame}

\section{Data Privacy}
\frame{\sectionpage}

\begin{frame}{Data Protection}
As a Data Scientist, you may be collecting Personal Information. If this data is not treated according to the law, you may be liable for significant fines. 
\begin{itemize}
\item Many countries have their own Data Protection laws
\item The EU General Data Protection Regulation is applicable to anyone physically located in the EU
\item Certain research communities require approval from ethics boards for data collection
\end{itemize}
\end{frame}

\begin{frame}{Data Protection}
Best Practices
\begin{itemize}
\item \textbf{Minimise} Data Collection
\item Be \textbf{transparent}; why are you collecting the data? Which data are you collecting? How will you share it? How long will you keep it?
\item Treat the data with \textbf{respect}; store it securely, anonymise it when possible
\item Make it clear how data owners can \textbf{retrieve} their data, or request \textbf{modification} or \textbf{deletion}?
\end{itemize}
\end{frame}

\begin{frame}{Anonymisation}
\begin{itemize}
\item Even if you anonymise the name, are individuals still identifiable from the data?
\item If you convert names to anonymous strings, can you get back to the name?  
\end{itemize}
\end{frame}

\begin{frame}{}
\begin{center}
\includegraphics[width=0.45\linewidth]{anonymous-form.png}
\end{center}
\end{frame}

\begin{frame}{Data Privacy - Summary}
\begin{itemize}
\item Minimise the collection of privacy impacting data 
\item Be transparent about data processing and transfer
\end{itemize}
\end{frame}

\begin{frame}{Questions?}
\begin{itemize}
\item Ask now
\item Find us during the break
\item You are welcome to contact us after the school
\end{itemize}
\end{frame}

\begin{frame}{Credits}
	\begin{itemize}
		\item Sebastian Lopienski (CERN IT) for security principles
		\item Stefan Lueders (CERN IT) for threats
        \item Hannah Short (CERN IT) for identity aspects 
	\end{itemize}
\end{frame}

\backcover

\end{document}